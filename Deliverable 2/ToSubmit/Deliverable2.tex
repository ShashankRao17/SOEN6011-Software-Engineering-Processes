\documentclass[12pt]{report}
\usepackage{graphicx}
\usepackage[margin=1in]{geometry}
\usepackage{wrapfig}
\usepackage{gensymb}
\usepackage{array}
\usepackage{hyperref}
\usepackage[utf8]{inputenc}
\usepackage[english]{babel}
\setlength{\parskip}{0.5em}

\begin{document}

\begin{center}
	\large\textbf{SOEN 6011:SOFTWARE ENGINEERING PROCESSES}\\
	\vspace{0.05cm}\large\textbf{\underline{Deliverable \#2}}\\
	\vspace{0.05cm}\small{SHASHANK RAO}\\
	\small\textbf SID:40104247\\

	\vspace{0.10cm}\small\textbf{https://github.com/ShashankRao17/SOEN6011-Software-Engineering-Processes}
	
\end{center}

\renewcommand \thesection{\arabic{section}}
\renewcommand \thesubsection{\arabic{section}.\arabic{subsection}}

\begin{center}
\section{Problem-4}
\end{center}	

\subsection{Debugger}

A debugger is a program that can examine the state of your program while your program is running.It is a computer program used by programmers to test and debug a target program. Debuggers may use instruction-set simulators, rather than running a program directly on the processor to achieve a higher level of control over its execution.

The debugger used for this project is the inbuilt JDT(Java Development Toolkit) debugger, present in Eclipse IDE.

\begin{itemize}
\item \textbf{\underline{Advantages:}}
	\begin{enumerate}
	\item[i.] One of the benefits that Eclipse provides is the ability to run code interactively by using its integrated debugger. Examining variables and expressions while executing code step-by-step is an invaluable tool for investigating problems with your code.
	\item[ii.] It often allows debugger-time altering of the variables and logic, which might be handy as well.
	\end{enumerate}
\item\textbf{\underline{Disadvantages:}}
	\begin{enumerate}
	\item[i.] Debugging in Eclipse sometimes needs different plug-ins which can lead to a plug-in-nightmare. Different plug-ins require different versions of the same requirement plug-in, and most often the installation order also makes a difference.
	\item[ii.] Many a times, the error messages that are encountered while debugging are inexplicable and may require more effort on the programmers part to understand and resolve/correct.\pagebreak
	\end{enumerate}
\end{itemize}

\subsection{Key quality attributes}

\begin{center}
	\begin{tabular}{|p{3.5cm}|p{10cm}|}
	\hline
	Attributes&Explaination pertaining to program\\
	\hline
	\vspace{0.1cm}\textbf{1. Correctness}&\vspace{0.1cm}When correct input(in the range of -1$\leq$x$\leq$1) is given by user, the program computes the value of inverse of Cosine(ArcCos) in terms of Radians(Rad) and Degrees	($^{\circ}$).\\
	\vspace{0.1cm}\textbf{2. Efficient}&\vspace{0.1cm}The program aims to compute the value of ArcCos of a value x given by the user in Radians as well as Degree. The output is correct for $\approx$ 2 decimal places for most of the cases and in some cases upto 4 decimal places and can be efficiently used any number of times without any hiccups in processing.\\
	\vspace{0.1cm}\textbf{3. Maintainable}&\vspace{0.1cm}To ensure greater accuracy in output, the value of n term used for 'Taylor Series Approximation' can be taken greater than 10. Such a change is a win-win scenario as it ensures better accuracy with also being less tedious to modify since its a separate function altogether with weak coupling with other defined functions.\\
	\vspace{0.1cm}\textbf{4. Robust}&\vspace{0.1cm}To ensure robustness of the program, appropriate(clear and helpful) error messages have been put into place to ensure correct input from users(pre-execution). The program also uses inbuilt error handling mechanisms from Java(try-catch block) which helps to cope with errors during execution as well.\\
	\vspace{0.1cm}\textbf{5. Usable}&\vspace{0.1cm}In terms of usability, the program provides an easy and interactive GUI for users to be able to communicate with the system. The program takes input from the user and also displays the results on the application interface thereby encapsulating the processing.\\
	\hline
	\end{tabular} 
\end{center}
\pagebreak

\subsection{The Static Code Analysis Tool - CheckStyle}
CheckStyle is a static code analysis tool used in software development for checking if Java source code complies with coding rules.
\begin{itemize}
\item \textbf{\underline{Advantages:}}
	\begin{enumerate}
	\item[i.] CheckStyle ensures that the code compiles with good programming practices such as Google Java Style,(Oracle/Sun Microsystems) Code Conventions for the Java Programming Language, etc.
	\item[ii.] Using ChechStyle may lead to improvement in the quality, readability, re-usability of code and also lead to reduced development cost in the long run.
	\end{enumerate}
\item \textbf{\underline{Disadvantages:}}
	\begin{enumerate}
	\item[i.] CheckStyle performs checks are limited mainly to presentation of the code and thus the correctness or completeness of the code cannot be confirmed.\\
	\end{enumerate}
	
\end{itemize}

\section{References}
\begin{enumerate}
\item[i.] https://stackoverflow.com/questions/25385173/what-is-a-debugger-and-how-can-it-help-me-diagnose-problems
\item[ii.] https://softwareengineering.stackexchange.com/questions/168540/what-are-the-advantages-of-using-the-java-debugger-over-println
\item[iii.] https://en.wikipedia.org/wiki/Checkstyle
\end{enumerate}

\end{document}