\documentclass[12pt]{report}
\usepackage{graphicx}
\usepackage[margin=1in]{geometry}
\usepackage{wrapfig}
\usepackage{gensymb}
\usepackage{array}
\usepackage{hyperref}
\usepackage[utf8]{inputenc}
\usepackage[english]{babel}
\setlength{\parskip}{0.5em}

\begin{document}

\begin{center}
	\large\textbf{SOEN 6011:SOFTWARE ENGINEERING PROCESSES}\\
	\vspace{0.05cm}\large\textbf{\underline{Deliverable \#2}}\\
	\vspace{0.05cm}\small{SHASHANK RAO}\\
	\small\textbf SID:40104247\\

	\vspace{0.10cm}\small\textbf https://github.com/ShashankRao17/SOEN6011-Software-Engineering-Processes 
	
\end{center}

\renewcommand \thesection{\arabic{section}}
\renewcommand \thesubsection{\arabic{section}.\arabic{subsection}}

\begin{center}
\section{Problem-4}
\end{center}	

\subsection{Debugger}
A debugger is a program that can examine the state of your program while your program is running. 
It is a computer program used by programmers to test and debug a target program. Debuggers may use instruction-set simulators, rather than running a program directly on the processor to achieve a higher level of control over its execution.\par
The debugger used for this project is the inbuilt JDT debugger, present in Eclipse IDE.

\begin{itemize}
\item \textbf{\underline{Advantages:}}
	\begin{enumerate}
	\item[i.] One of the benefits that Eclipse provides is the ability to run code interactively by using its integrated debugger. Examining variables and expressions while executing code step-by-step is an invaluable tool for investigating problems with your code.
	\item[ii.] It often allows debugger-time altering of the variables and logic, which might be handy as well.
	\end{enumerate}
\item\textbf{\underline{Disadvantages:}}
\end{itemize}

\end{document}