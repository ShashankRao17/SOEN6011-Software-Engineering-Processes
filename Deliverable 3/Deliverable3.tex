\documentclass[12pt]{report}
\usepackage{graphicx}
\usepackage[margin=1in]{geometry}
\usepackage{wrapfig}
\usepackage{gensymb}
\usepackage{array}
\usepackage{hyperref}
\usepackage[utf8]{inputenc}
\usepackage[english]{babel}
\setlength{\parskip}{0.5em}

\begin{document}

\begin{center}
	\large\textbf{SOEN 6011:SOFTWARE ENGINEERING PROCESSES}\\
	\vspace{0.05cm}\large\textbf{\underline{Deliverable \#2}}\\
	\vspace{0.05cm}\small{SHASHANK RAO}\\
	\small\textbf SID:40104247\\

	\vspace{0.10cm}\small\textbf{https://github.com/ShashankRao17/SOEN6011-Software-Engineering-Processes}
	
\end{center}

\renewcommand \thesection{\arabic{section}}
\renewcommand \thesubsection{\arabic{section}.\arabic{subsection}}

\begin{center}
	\section{Problem-5}
\end{center}

\subsection{Source code review for Function F2 - tan(x)}
%	\begin{itemize}
	\subsubsection{Objective}The objective of this code review is to ensure the supplied code for the function 'tan(x)' complies with the mentioned coding standards and guidelines given by the developer, to ensure better code quality if possible and to check for defects in the supplied code on the basis of correctness, efficiency, maintainability, usability and robustness.
	\subsubsection{Code Review Approach}
	\begin{enumerate}
		\item[1.] To ensure transparency  and fairness in the review process, an approach mentioned by George T. Doran in the November 1981 issue of \textit{Management Review} called "There's a S.M.A.R.T. way to write management's goals and objectives" was used. This approach(also called the \textit{S.M.A.R.T.} criteria) gives a set of objectives and goals to ensure the code review is carried out in an established manner. As per the paper, S.M.A.R.T. means the following(although over the years, the acronym has taken many different forms):
	\begin{itemize}
	\item[•]\textbf{S}pecific - an area of improvement.
	\item[•]\textbf{M}easurable - quantify or suggest an indicator of progress.
	\item[•]\textbf{A}ssignable - specify who will do it.
	\item[•]\textbf{R}ealistic - state results can be realistically achieved, given the available resources.  
	\item[•]\textbf{T}ime Related - when can the results be achieved.
	\end{itemize}
	(It should be noted that not all of the S.M.A.R.T. techniques need to be quantified for a successful review.)
		\item[2.] The coding review is also done using static code analysis plugins(CheckStyle,PMD) which gives a good sense of judgement on the various industry coding standards available and the ones used by the developer for his/her function and how they comply with them.
    \end{enumerate}		
	\subsubsection{Coding Style}The coding style followed here is the well renowned Google Checks coding standard. This fact is evident by the use of CheckStyle plugin which maintains either the Google Checks configuration or the Sun Checks configuration. The developer has configured Google Checks as the default configuration during the development process.\par
	A mix of S.M.A.R.T. criteria and key quality attribute perspective, the following points are worth noting:
	\begin{center}
	\begin{tabular}{|p{5.2cm}|p{10cm}|}
	\hline
	Criteria&Explaination\\
	\hline
	\vspace{0.1cm}\textbf{Specific}&\vspace{0.1cm} A particular area of improvement could be to differentiate the program logic from the main class thereby ensuring encapsulation. Additionally the variable declarations could be improved to match the coding standards set by the team.\\
	\vspace{0.12cm}\textbf{Assignable(Maintainable)}&\vspace{0.1cm}In a scenario where another developer has to go through or contribute towards the development of this function, ease of maintenance such as low coupling(since there is only a since main function) is provided along with clear JavaDoc documentation.\\
	\vspace{0.12cm}\textbf{Realistic(Correctness)}&\vspace{0.1cm}The system was developed without using any in-built Math function and has been implemented using the Taylor series and reduction formula and hence the results are close to accurate and realistically achievable as they plain and simple calculations and no approximations whatsoever.\\
	\vspace{0.05cm}\textbf{Time-Related(Efficient)}&\vspace{0.05cm} The program is developed keeping correctness and efficiency in mind. The system responds to users' input in quick time since most of the coefficients needed for Taylor series approximation are defined as static variables thus reducing the overall execution time and CPU utilization in the long run.\\
	\vspace{0.05cm}\textbf{Usability \& Robustness}&\vspace{0.05cm} The program interacts with the user using a Textual User Interface(TUI) with appropriate error handling mechanisms and messages in place to ensure robustness.\\ 
	\hline
	\end{tabular}
	\end{center}
	\subsubsection{Conclusion}To conclude, other than the point on encapsulation, the overall program is well-structured to encompass all the key attributes keeping the user at the center of its functioning.
%	\end{itemize}
\begin{center}
	\section{Problem-7}
\end{center}

\subsection{Test review for Function F3 - sinh(x)}
\subsubsection{Objective}
\subsubsection{Testing Approach and Computing environment}


\pagebreak\textbf{References}
	\begin{enumerate}
	\item[i.]https://en.wikipedia.org/wiki/SMART\_criteria
	\item[ii.]https://nyu-cds.github.io/effective-code-reviews/02-best-practices/
	\item[iii.]http://thinkapps.com/blog/development/what-is-code-review/
	\item[iv.] Doran, G. T. (1981). "There's a S.M.A.R.T. way to write management's goals and objectives". Management Review. 70 (11): 35–36.
	\end{enumerate}

\end{document}